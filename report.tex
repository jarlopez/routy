\documentclass[a4paper, 11pt]{article}
\usepackage[utf8]{inputenc}
\usepackage{graphicx}
\usepackage{url}
\usepackage{hyperref}

%opening
\title{Seminar Report: Routy -- A Small Routing Protocol}
\author{Johan Mickos \\ johanmi@kth.se}
\date{\today{}}

\begin{document}

\maketitle
\newpage

\section{Introduction}

• describe the structure of a link-state routing protocol
    "Link-state algorithms are sometimes characterized informally as each router, "telling the world about its neighbours."
• describe how a consistent view is maintained
    Each router informs the network about its neighbors
• reflect on the problems related to network failures
    Packets get dropped when routers become disconnected
    Erlang allows for monitoring other processes. As a result, a router who has a direct link to another will know when the router drops from the network (according to Erlang VM's rules and definitions)

\section{Main Problems and Solutions}
\subsection{Security Issues}
Rogue router
\subsection{Direct Routing to Neighbors}
As the assignment is presented, there is no way for a router to know how to route to its immediate neighbors after they have been added to its list of interfaces.

This is due to the fact that adding a new unidirectional link \textit{does not automatically} add the new node to the map and thus the routing table is never updated with the direct link.

This is solved by allowing each router to update its \texttt{Map} with the fact that it (\texttt{Name}) can reach the new node (\texttt{Node}):
\begin{verbatim}
router(Name, N, Hist, Intf, Table, Map) ->
    receive
        {add, Node, Pid} ->
            Ref = erlang:monitor(process,Pid),
            Intf1 = intf:add(Node, Ref, Pid, Intf),

            % Also update table
            Map1 = map:update(Name, [Node], Map),
            Table1 = dijkstra:table(intf:list(Intf1), Map1),
            router(Name, N + 1, Hist, Intf1, Table1, Map1);
        % ...
    end.
\end{verbatim}

\subsection{Dropping from the Network}
When a node is dropped from the network, its neighbors are alerted (thanks to Erlang).

Restoring the node, however, is far more involved. This is because its neighbors maintain a history of the dropped node's messages. When a node restarts, its internal counter will be _lower_ than the counters of its neighbors (assuming the network has been "alive" for some time). Thus, any broadcasts being sent out from the previously dropped node will be ignore _until_ the node's internal counter exceeds the counters at its neighbors. To mitigate this, neighbors can reset or remove the history entries from dropped nodes:

\begin{verbatim}
-module(hist).
drop(Node, History) ->
    lists:keydelete(Node, 1, History).
\end{verbatim}

\begin{verbatim}
-module(hist).
drop(Node, History) ->
    lists:keydelete(Node, 1, History).
\end{verbatim}

\begin{verbatim}
{'DOWN', Ref, process, _, _} ->
    % ...
    Hist1 = hist:drop(Down, Hist),

    % Update link-state
    Message = {links, Name, N, intf:list(Intf1)},
    intf:broadcast(Message, Intf1),
\end{verbatim}

When a node drops from the network, its alerted neighbors will forget its history. Then, if/when the node rejoins the network and re-establishes its links, its old neighbors will be able to respond to the new link states.

One edge case this does not address is when the dropped node establishes links to _new_ neighbors, who may still have a higher history counter for the given node. They will continue ignoring the node until its counter is greater than its previous history counter.

\section{Evaluation}
\section{Conclusion}
\end{document}
